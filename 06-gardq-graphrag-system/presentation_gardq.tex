\documentclass[10pt, aspectratio=169]{beamer}
\usetheme{metropolis}
\usepackage[utf8]{inputenc}
\usepackage[french]{babel}
\usepackage{tikz}
\usepackage{pgfplots}
\usepackage{listings}
\usepackage{xcolor}
\usepackage{fontawesome}
\usepackage{graphicx}
\usepackage{array}
\usepackage{tabularx}
\usepackage{booktabs}
\usepackage{multicol}

% TikZ libraries
\usetikzlibrary{shapes.geometric, arrows.meta, positioning, shadows, patterns, decorations.pathreplacing, calc, fit, backgrounds}
\pgfplotsset{compat=1.17}

% Couleurs personnalisées
\definecolor{gardqblue}{RGB}{79, 195, 247}
\definecolor{gardqgreen}{RGB}{129, 199, 132}
\definecolor{gardqorange}{RGB}{255, 183, 77}
\definecolor{gardqred}{RGB}{239, 83, 80}
\definecolor{gardqpurple}{RGB}{149, 117, 205}
\definecolor{gardqgray}{RGB}{117, 117, 117}
\definecolor{neo4jgreen}{RGB}{0, 140, 201}

% Configuration des listings
\lstset{
    basicstyle=\tiny\ttfamily,
    keywordstyle=\color{gardqblue}\bfseries,
    commentstyle=\color{gardqgray}\itshape,
    stringstyle=\color{gardqgreen},
    breaklines=true,
    showstringspaces=false,
    frame=single,
    frameround=tttt,
    framexleftmargin=0pt,
    numbers=left,
    numberstyle=\tiny\color{gardqgray},
    xleftmargin=0.5cm,
    xrightmargin=0.2cm
}

% Informations de la présentation
\title{GARDQ}
\subtitle{Graph Augmented Retrieval for Data Quality}
\author{Ibrahim Adiao}
\date{\today}
\institute{Système Intelligent de Gestion des Incidents IT}

% Styles pour les nœuds
\tikzstyle{ticketnode} = [rectangle, rounded corners, minimum width=3cm, minimum height=1cm, text centered, draw=black, fill=gardqblue!20, font=\footnotesize]
\tikzstyle{sectionnode} = [rectangle, rounded corners, minimum width=2.5cm, minimum height=0.8cm, text centered, draw=black, fill=gardqgreen!20, font=\footnotesize]
\tikzstyle{entitynode} = [rectangle, rounded corners, minimum width=2cm, minimum height=0.8cm, text centered, draw=black, fill=gardqorange!20, font=\footnotesize]
\tikzstyle{processnode} = [rectangle, rounded corners, minimum width=3cm, minimum height=1cm, text centered, draw=black, fill=gardqpurple!20, font=\footnotesize]
\tikzstyle{arrow} = [thick,->,>=stealth]

\begin{document}

% Page de titre
\begin{frame}
    \titlepage
\end{frame}

% Table des matières
\begin{frame}{Agenda}
    \tableofcontents
\end{frame}

% Section 1: Introduction
\section{Introduction et Problématique}

\begin{frame}{Contexte: Gestion des Incidents IT}
    \begin{columns}[T]
        \begin{column}{0.5\textwidth}
            \begin{block}{Défis Actuels}
                \begin{itemize}
                    \item \faExclamationTriangle{} Volume croissant d'incidents
                    \item \faClock{} Temps de résolution élevé
                    \item \faDatabase{} Données non structurées
                    \item \faUsers{} Expertise dispersée
                \end{itemize}
            \end{block}
        \end{column}
        \begin{column}{0.5\textwidth}
            \begin{block}{Solution GARDQ}
                \begin{itemize}
                    \item \faLightbulb{} GraphRAG innovant
                    \item \faBrain{} IA générative (GPT-4)
                    \item \faProjectDiagram{} Neo4j Knowledge Graph
                    \item \faChartLine{} Amélioration continue
                \end{itemize}
            \end{block}
        \end{column}
    \end{columns}
    
    \vspace{0.5cm}
    \begin{alertblock}{Objectif}
        Transformer l'historique des incidents en base de connaissances intelligente pour améliorer la qualité des données et réduire le temps de résolution.
    \end{alertblock}
\end{frame}

\begin{frame}{Vue d'Ensemble du Système GARDQ}
    \begin{center}
    \begin{tikzpicture}[scale=0.8, transform shape]
        % Nodes
        \node[ticketnode] (ticket) at (0,0) {Ticket\\d'Incident};
        \node[processnode] (construct) at (3,0) {Construction\\du Graphe};
        \node[processnode] (neo4j) at (6,0) {\includegraphics[width=0.5cm]{neo4j-icon.png}\\Neo4j};
        \node[processnode] (retrieval) at (9,0) {Retrieval\\Augmenté};
        \node[processnode] (llm) at (6,-3) {\faRobot{} GPT-4o\\mini};
        \node[ticketnode] (solution) at (9,-3) {Solution\\Générée};
        
        % Arrows
        \draw[arrow] (ticket) -- (construct);
        \draw[arrow] (construct) -- (neo4j);
        \draw[arrow] (neo4j) -- (retrieval);
        \draw[arrow] (construct) -- (llm);
        \draw[arrow] (retrieval) -- (solution);
        \draw[arrow] (llm) -- (solution);
        
        % Background
        \begin{scope}[on background layer]
            \node[fit=(ticket)(construct)(neo4j)(retrieval)(llm)(solution), 
                  fill=gray!10, rounded corners, inner sep=0.3cm] {};
        \end{scope}
    \end{tikzpicture}
    \end{center}
    
    \vspace{0.5cm}
    \begin{enumerate}
        \item \textbf{Analyse} et structuration des tickets via LLM
        \item \textbf{Construction} du graphe de connaissances
        \item \textbf{Recherche} augmentée par similarité
        \item \textbf{Génération} de solutions contextualisées
    \end{enumerate}
\end{frame}

% Section 2: Construction du Graphe
\section{Construction du Graphe de Connaissances}

\begin{frame}{Transformation: Du Texte au Graphe}
    \begin{columns}[T]
        \begin{column}{0.45\textwidth}
            \begin{block}{Données Non Structurées}
                \begin{tcolorbox}[colback=red!5!white,colframe=red!75!black]
                    \footnotesize
                    \textbf{Ticket INC0001234:}\\
                    "Problème de connexion à l'application ServiceNow. Erreur 500. Priorité P2. Résolu en redémarrant le service."
                \end{tcolorbox}
            \end{block}
        \end{column}
        
        \begin{column}{0.55\textwidth}
            \begin{block}{Graphe Structuré}
                \begin{center}
                \begin{tikzpicture}[scale=0.6, transform shape]
                    % Central ticket
                    \node[ticketnode, minimum width=2cm] (ticket) at (0,0) {Ticket\\INC0001234};
                    
                    % Sections
                    \node[sectionnode, minimum width=1.8cm] (summary) at (-2,-2) {SUMMARY};
                    \node[sectionnode, minimum width=1.8cm] (priority) at (0,-2) {PRIORITY};
                    \node[sectionnode, minimum width=1.8cm] (solution) at (2,-2) {SOLUTION};
                    
                    % Entities
                    \node[entitynode, minimum width=1.5cm] (p2) at (0,-4) {Priority\\P2};
                    \node[entitynode, minimum width=1.5cm] (app) at (-3,-4) {Application\\ServiceNow};
                    
                    % Relations
                    \draw[arrow] (ticket) -- node[left, font=\tiny] {HAS\_SECTION} (summary);
                    \draw[arrow] (ticket) -- node[right, font=\tiny] {HAS\_SECTION} (priority);
                    \draw[arrow] (ticket) -- node[right, font=\tiny] {HAS\_SECTION} (solution);
                    \draw[arrow] (priority) -- node[right, font=\tiny] {REFERS\_TO} (p2);
                    \draw[arrow, dashed] (summary) -- node[left, font=\tiny] {REFERS\_TO} (app);
                \end{tikzpicture}
                \end{center}
            \end{block}
        \end{column}
    \end{columns}
\end{frame}

\begin{frame}[fragile]{Pipeline de Transformation avec LLM}
    \begin{center}
    \begin{tikzpicture}[scale=0.9, transform shape, node distance=1.5cm]
        % Process flow
        \node[processnode] (raw) at (0,0) {Ticket Brut};
        \node[processnode, right=of raw] (analyze) {Analyse GPT-4};
        \node[processnode, right=of analyze] (extract) {Extraction};
        \node[processnode, below=of extract] (embed) {Embeddings};
        \node[processnode, left=of embed] (entities) {Entités};
        \node[processnode, below=of embed] (store) {Stockage Neo4j};
        
        % Arrows with labels
        \draw[arrow] (raw) -- (analyze);
        \draw[arrow] (analyze) -- (extract);
        \draw[arrow] (extract) -- (embed);
        \draw[arrow] (extract) -- (entities);
        \draw[arrow] (embed) -- (store);
        \draw[arrow] (entities) -- (store);
        
        % Annotations
        \node[above=0.3cm of analyze, font=\tiny, text=gardqgray] {Structuration};
        \node[right=0.3cm of extract, font=\tiny, text=gardqgray] {Sections};
        \node[below=0.3cm of entities, font=\tiny, text=gardqgray] {Normalisation};
    \end{tikzpicture}
    \end{center}
    
    \vspace{0.3cm}
    \begin{columns}[T]
        \begin{column}{0.5\textwidth}
            \begin{lstlisting}[language=Python, basicstyle=\tiny\ttfamily]
# Analyse avec GPT-4
structured = llm.analyze_ticket(raw_ticket)

# Extraction des sections
sections = {
    "SUMMARY": structured["summary"],
    "SOLUTION": structured["solution"],
    "ROOT_CAUSE": structured["root_cause"]
}
            \end{lstlisting}
        \end{column}
        \begin{column}{0.5\textwidth}
            \begin{lstlisting}[language=Python, basicstyle=\tiny\ttfamily]
# Création des embeddings
for section in ["SUMMARY", "SOLUTION"]:
    embedding = encoder.encode(sections[section])
    
# Extraction des entités
entities = {
    "Application": structured["app"],
    "Priority": structured["priority"]
}
            \end{lstlisting}
        \end{column}
    \end{columns}
\end{frame}

\begin{frame}{Types de Relations dans GARDQ}
    \begin{center}
    \begin{tikzpicture}[scale=0.8, transform shape]
        % Define nodes in a circular arrangement
        \node[ticketnode] (t1) at (0,0) {Ticket A};
        \node[ticketnode] (t2) at (4,0) {Ticket B};
        \node[ticketnode] (t3) at (2,2) {Ticket Parent};
        \node[sectionnode] (s1) at (-2,-1) {Section};
        \node[entitynode] (e1) at (-2,-3) {Entity};
        
        % Relations
        \draw[arrow, thick, color=gardqblue] (t1) -- node[below] {\footnotesize HAS\_SECTION} (s1);
        \draw[arrow, thick, color=gardqgreen] (s1) -- node[left] {\footnotesize REFERS\_TO} (e1);
        \draw[arrow, thick, color=gardqorange] (t1) -- node[above left] {\footnotesize PARENT\_TICKET} (t3);
        \draw[arrow, thick, color=gardqred, <->] (t1) -- node[below] {\footnotesize SIMILAR\_TO} (t2);
        
        % Similarity score
        \node[above=0.1cm of t1, font=\tiny] {sim: 0.85};
    \end{tikzpicture}
    \end{center}
    
    \vspace{0.5cm}
    \begin{table}[h]
    \centering
    \footnotesize
    \begin{tabular}{lll}
    \toprule
    \textbf{Relation} & \textbf{Direction} & \textbf{Description} \\
    \midrule
    HAS\_SECTION & Ticket → Section & Contenu structuré \\
    REFERS\_TO & Section → Entity & Références métier \\
    PARENT\_TICKET & Child → Parent & Hiérarchie \\
    SIMILAR\_TO & Bidirectionnelle & Similarité > 0.7 \\
    \bottomrule
    \end{tabular}
    \end{table}
\end{frame}

% Section 3: Stockage Neo4j
\section{Stockage dans Neo4j}

\begin{frame}{Architecture de Stockage Neo4j}
    \begin{center}
    \begin{tikzpicture}[scale=0.9, transform shape]
        % Neo4j logo/database
        \node[cylinder, draw, fill=neo4jgreen!20, minimum height=2cm, minimum width=3cm] (db) at (0,0) {Neo4j\\Knowledge Graph};
        
        % Node types around the database
        \node[ticketnode, above left=1.5cm of db] (ticket) {Ticket\\ticket\_id};
        \node[sectionnode, above right=1.5cm of db] (section) {Section\\content, embedding};
        \node[entitynode, below left=1.5cm of db] (cause) {Cause\\name};
        \node[entitynode, below right=1.5cm of db] (priority) {Priority\\name};
        
        % Connections
        \draw[arrow, dashed] (ticket) -- (db);
        \draw[arrow, dashed] (section) -- (db);
        \draw[arrow, dashed] (cause) -- (db);
        \draw[arrow, dashed] (priority) -- (db);
        
        % Constraints box
        \node[rectangle, draw, fill=yellow!20, below=2.5cm of db, text width=6cm, font=\tiny] (constraints) {
            \textbf{Contraintes:}\\
            • UNIQUE (Ticket.ticket\_id)\\
            • UNIQUE (Section.section\_id, Section.ticket\_id)\\
            • UNIQUE (Cause.name), UNIQUE (Priority.name)
        };
    \end{tikzpicture}
    \end{center}
    
    \begin{block}{Caractéristiques du Modèle}
        \begin{itemize}
            \item \faCheck{} Séparation contenu/structure
            \item \faCheck{} Embeddings vectoriels (384 dimensions)
            \item \faCheck{} Entités partagées entre tickets
            \item \faCheck{} Index optimisés pour la performance
        \end{itemize}
    \end{block}
\end{frame}

\begin{frame}[fragile]{Requêtes Cypher - Création d'un Ticket}
    \begin{lstlisting}[language=SQL, basicstyle=\scriptsize\ttfamily, keywordstyle=\color{neo4jgreen}\bfseries]
// 1. Créer le nœud Ticket
MERGE (t:Ticket {ticket_id: 'INC0001234'})

// 2. Créer les sections avec embeddings
CREATE (s:Section {
    section_id: 'INC0001234_SUMMARY',
    type: 'SUMMARY',
    content: 'Problème de connexion ServiceNow',
    ticket_id: 'INC0001234',
    embedding: [0.123, -0.456, 0.789, ...] // 384 dimensions
})

// 3. Relier ticket et sections
MATCH (t:Ticket {ticket_id: 'INC0001234'})
MATCH (s:Section {ticket_id: 'INC0001234'})
MERGE (t)-[:HAS_SECTION]->(s)

// 4. Créer et relier les entités
MERGE (app:Application {name: 'ServiceNow'})
MERGE (prio:Priority {name: 'P2'})
WITH app, prio
MATCH (s_app:Section {section_id: 'INC0001234_APPLICATION'})
MATCH (s_prio:Section {section_id: 'INC0001234_PRIORITY'})
MERGE (s_app)-[:REFERS_TO]->(app)
MERGE (s_prio)-[:REFERS_TO]->(prio)
    \end{lstlisting}
\end{frame}

\begin{frame}{Optimisations et Performance}
    \begin{columns}[T]
        \begin{column}{0.5\textwidth}
            \begin{block}{Index Neo4j}
                \begin{lstlisting}[language=SQL, basicstyle=\tiny\ttfamily]
CREATE INDEX ticket_idx 
  FOR (t:Ticket) ON (t.ticket_id);
  
CREATE INDEX section_idx 
  FOR (s:Section) 
  ON (s.section_id, s.ticket_id);
  
CREATE INDEX section_type_idx 
  FOR (s:Section) ON (s.type);
                \end{lstlisting}
            \end{block}
        \end{column}
        
        \begin{column}{0.5\textwidth}
            \begin{block}{Métriques de Performance}
                \begin{center}
                \begin{tikzpicture}[scale=0.7]
                    \begin{axis}[
                        ybar,
                        bar width=0.5cm,
                        ylabel={\footnotesize Performance},
                        symbolic x coords={Requête,Insertion,Similarité},
                        xtick=data,
                        ymin=0,
                        ymax=150,
                        ylabel near ticks,
                        xlabel near ticks,
                        tick label style={font=\tiny},
                        label style={font=\footnotesize},
                        height=4cm,
                        width=6cm
                    ]
                    \addplot[fill=gardqblue] coordinates {
                        (Requête,50)
                        (Insertion,100)
                        (Similarité,80)
                    };
                    \end{axis}
                \end{tikzpicture}
                \end{center}
                \footnotesize
                \faCheckCircle{} Temps moyen < 100ms\\
                \faDatabase{} Capacité > 1M tickets\\
                \faLink{} 15-20 relations/ticket
            \end{block}
        \end{column}
    \end{columns}
\end{frame}

% Section 4: Processus de Retrieval
\section{Processus de Retrieval Augmenté}

\begin{frame}{Retrieval en 3 Étapes}
    \begin{center}
    \begin{tikzpicture}[scale=0.85, transform shape]
        % Step boxes
        \node[processnode, fill=gardqblue!30, minimum width=8cm] (step1) at (0,0) {
            \textbf{1. Analyse de la Requête}\\
            \footnotesize Extraction d'entités et intentions via GPT-4
        };
        
        \node[processnode, fill=gardqgreen!30, minimum width=8cm, below=0.8cm of step1] (step2) {
            \textbf{2. Recherche de Tickets Similaires}\\
            \footnotesize Similarité cosinus sur embeddings vectoriels
        };
        
        \node[processnode, fill=gardqorange!30, minimum width=8cm, below=0.8cm of step2] (step3) {
            \textbf{3. Extraction du Sous-Graphe}\\
            \footnotesize Requêtes Cypher dynamiques pour contexte enrichi
        };
        
        % Arrows
        \draw[arrow, very thick] (step1) -- (step2);
        \draw[arrow, very thick] (step2) -- (step3);
        
        % Side annotations
        \node[right=0.5cm of step1, text width=3cm, font=\tiny] {
            • Entités métier\\
            • Type de problème\\
            • Mots-clés
        };
        
        \node[right=0.5cm of step2, text width=3cm, font=\tiny] {
            • Top-K = 10\\
            • Seuil > 0.5\\
            • Multi-sections
        };
        
        \node[right=0.5cm of step3, text width=3cm, font=\tiny] {
            • Relations 2-hop\\
            • Entités liées\\
            • Historique
        };
    \end{tikzpicture}
    \end{center}
\end{frame}

\begin{frame}[fragile]{Étape 1: Analyse de la Requête}
    \begin{columns}[T]
        \begin{column}{0.5\textwidth}
            \begin{block}{Requête Utilisateur}
                \begin{tcolorbox}[colback=blue!5!white,colframe=blue!75!black]
                    \footnotesize
                    "Impossible de se connecter à ServiceNow, erreur 500"
                \end{tcolorbox}
            \end{block}
            
            \begin{block}{Prompt GPT-4}
                \begin{lstlisting}[language=Python, basicstyle=\tiny\ttfamily]
prompt = f"""
Analysez cette requête d'incident:
"{query}"

Extrayez:
1. Entités (applications, services)
2. Type de problème
3. Mots-clés importants
4. Intention de recherche
"""
                \end{lstlisting}
            \end{block}
        \end{column}
        
        \begin{column}{0.5\textwidth}
            \begin{block}{Résultat de l'Analyse}
                \begin{lstlisting}[language=JavaScript, basicstyle=\tiny\ttfamily]
{
  "entities": [
    "ServiceNow", 
    "login"
  ],
  "problem_type": "authentication",
  "keywords": [
    "connexion", 
    "erreur 500"
  ],
  "intent": "find_similar_auth_issues"
}
                \end{lstlisting}
            \end{block}
            
            \begin{alertblock}{Avantage}
                \footnotesize
                Compréhension contextuelle permettant une recherche ciblée
            \end{alertblock}
        \end{column}
    \end{columns}
\end{frame}

\begin{frame}[fragile]{Étape 2: Recherche par Similarité}
    \begin{center}
    \begin{tikzpicture}[scale=0.8, transform shape]
        % Query embedding
        \node[rectangle, draw, fill=gardqpurple!20] (query) at (0,0) {Query\\Embedding};
        
        % Database
        \node[cylinder, draw, fill=neo4jgreen!20, right=2cm of query] (db) {Neo4j};
        
        % Similar tickets
        \node[ticketnode, above right=1cm and 3cm of db] (t1) {Ticket 1\\sim: 0.92};
        \node[ticketnode, right=3cm of db] (t2) {Ticket 2\\sim: 0.87};
        \node[ticketnode, below right=1cm and 3cm of db] (t3) {Ticket 3\\sim: 0.84};
        
        % Arrows
        \draw[arrow, thick] (query) -- node[above] {\tiny Cosine} (db);
        \draw[arrow, dashed] (db) -- (t1);
        \draw[arrow, dashed] (db) -- (t2);
        \draw[arrow, dashed] (db) -- (t3);
    \end{tikzpicture}
    \end{center}
    
    \begin{lstlisting}[language=SQL, basicstyle=\scriptsize\ttfamily]
MATCH (t:Ticket)-[:HAS_SECTION]->(s:Section)
WHERE s.embedding IS NOT NULL
WITH t, s, gds.similarity.cosine(s.embedding, $query_embedding) AS similarity
WHERE similarity > 0.5
WITH t, MAX(similarity) AS max_sim
ORDER BY max_sim DESC
LIMIT 10
RETURN t.ticket_id, max_sim
    \end{lstlisting}
\end{frame}

\begin{frame}[fragile]{Étape 3: Extraction du Sous-Graphe}
    \begin{lstlisting}[language=SQL, basicstyle=\tiny\ttfamily, keywordstyle=\color{neo4jgreen}\bfseries]
// Requête Cypher générée dynamiquement
MATCH (t:Ticket)
WHERE t.ticket_id IN $similar_ticket_ids
MATCH path = (t)-[:HAS_SECTION]->(s:Section)
OPTIONAL MATCH (s)-[:REFERS_TO]->(e)
WHERE e:Application AND e.name = 'ServiceNow'
OPTIONAL MATCH (t)-[:PARENT_TICKET]->(parent:Ticket)
OPTIONAL MATCH (t)-[:SIMILAR_TO]-(related:Ticket)
RETURN 
    t.ticket_id as ticket,
    collect(DISTINCT {
        section: s.type,
        content: s.content,
        entity: e.name
    }) as details,
    parent.ticket_id as parent,
    collect(DISTINCT related.ticket_id) as related_tickets
    \end{lstlisting}
    
    \begin{block}{Contexte Enrichi Extrait}
        \begin{itemize}
            \item \faCheckSquare{} Sections pertinentes des tickets similaires
            \item \faCheckSquare{} Entités métier communes
            \item \faCheckSquare{} Relations parent-enfant
            \item \faCheckSquare{} Réseau de tickets liés
        \end{itemize}
    \end{block}
\end{frame}

% Section 5: Communication avec le LLM
\section{Communication avec le LLM}

\begin{frame}{Architecture de Communication avec GPT-4}
    \begin{center}
    \begin{tikzpicture}[scale=0.85, transform shape, node distance=2cm]
        % Input components
        \node[rectangle, draw, fill=gardqblue!20, text width=2.5cm, align=center] (context) at (0,0) {Contexte\\Graphe};
        \node[rectangle, draw, fill=gardqgreen!20, text width=2.5cm, align=center, below=0.5cm of context] (query) {Requête\\Utilisateur};
        \node[rectangle, draw, fill=gardqorange!20, text width=2.5cm, align=center, above=0.5cm of context] (system) {Instructions\\Système};
        
        % Central processing
        \node[processnode, right=2cm of context, minimum width=3cm] (prompt) {Prompt\\Engineering};
        
        % API
        \node[processnode, fill=gardqpurple!30, right=2cm of prompt, minimum width=3cm] (api) {\faRobot{} GPT-4o\\mini API};
        
        % Output
        \node[rectangle, draw, fill=gardqgreen!30, right=2cm of api, text width=2.5cm, align=center] (output) {Solution\\Structurée};
        
        % Connections
        \draw[arrow] (context) -- (prompt);
        \draw[arrow] (query) -- (prompt);
        \draw[arrow] (system) -- (prompt);
        \draw[arrow] (prompt) -- node[above] {\tiny Prompt} (api);
        \draw[arrow] (api) -- node[above] {\tiny JSON} (output);
        
        % Annotations
        \node[below=0.3cm of prompt, font=\tiny, text=gardqgray] {Optimisation};
        \node[below=0.3cm of api, font=\tiny, text=gardqgray] {Génération};
    \end{tikzpicture}
    \end{center}
    
    \vspace{0.3cm}
    \begin{columns}[T]
        \begin{column}{0.5\textwidth}
            \begin{block}{Composants du Prompt}
                \footnotesize
                \faCheckCircle{} Instructions système\\
                \faCheckCircle{} Contexte enrichi du graphe\\
                \faCheckCircle{} Requête utilisateur\\
                \faCheckCircle{} Format de sortie structuré
            \end{block}
        \end{column}
        \begin{column}{0.5\textwidth}
            \begin{block}{Optimisations}
                \footnotesize
                \faRocket{} Few-shot examples\\
                \faRocket{} Température = 0.3\\
                \faRocket{} Max tokens optimisé\\
                \faRocket{} Retry avec clarification
            \end{block}
        \end{column}
    \end{columns}
\end{frame}

\begin{frame}[fragile]{Prompt Engineering Optimisé}
    \begin{lstlisting}[language=Python, basicstyle=\scriptsize\ttfamily]
def build_llm_prompt(query: str, graph_context: dict) -> str:
    system_prompt = """
    Vous êtes un expert en résolution d'incidents IT.
    Basez-vous sur les tickets similaires pour suggérer une solution.
    
    Format de réponse:
    1. Analyse du problème
    2. Solution suggérée (étapes détaillées)
    3. Tickets de référence utilisés
    4. Niveau de confiance (0-100%)
    """
    
    context = f"""
    Tickets similaires trouvés:
    {format_tickets(graph_context['similar_tickets'])}
    
    Entités communes:
    - Applications: {graph_context['applications']}
    - Causes fréquentes: {graph_context['common_causes']}
    
    Solutions appliquées précédemment:
    {format_solutions(graph_context['solutions'])}
    """
    
    return f"{system_prompt}\n\n{context}\n\nProblème: {query}"
    \end{lstlisting}
\end{frame}

\begin{frame}[fragile]{Exemple de Réponse Générée}
    \begin{block}{Entrée}
        \footnotesize
        "Impossible de se connecter à ServiceNow, erreur 500"
    \end{block}
    
    \begin{block}{Sortie GPT-4}
        \begin{lstlisting}[language=JavaScript, basicstyle=\tiny\ttfamily]
{
  "analysis": "Problème d'authentification ServiceNow avec erreur serveur",
  "suggested_solution": {
    "steps": [
      "1. Vérifier le statut du service ServiceNow",
      "2. Contrôler les logs d'authentification",
      "3. Redémarrer le service d'authentification SSO",
      "4. Vider le cache des sessions utilisateurs"
    ],
    "estimated_time": "15-20 minutes"
  },
  "reference_tickets": ["INC0001234", "INC0001567", "INC0001890"],
  "confidence": 85,
  "additional_notes": "Pattern récurrent après mise à jour système"
}
        \end{lstlisting}
    \end{block}
    
    \begin{alertblock}{Points Clés}
        \footnotesize
        \faCheck{} Solution contextualisée basée sur l'historique\\
        \faCheck{} Références traçables aux tickets sources\\
        \faCheck{} Score de confiance pour la transparence
    \end{alertblock}
\end{frame}

% Section 6: Résultats et Performance
\section{Résultats et Performance}

\begin{frame}{Métriques de Performance}
    \begin{columns}[T]
        \begin{column}{0.5\textwidth}
            \begin{block}{Métriques de Récupération}
                \begin{center}
                \begin{tikzpicture}[scale=0.7]
                    \begin{axis}[
                        ybar,
                        bar width=0.4cm,
                        ylabel={\footnotesize Score},
                        symbolic x coords={MRR,Recall@5,NDCG@10},
                        xtick=data,
                        ymin=0,
                        ymax=1,
                        legend pos=north west,
                        tick label style={font=\tiny},
                        label style={font=\footnotesize},
                        height=5cm,
                        width=6cm
                    ]
                    \addplot[fill=gardqred!70] coordinates {
                        (MRR,0.42) (Recall@5,0.58) (NDCG@10,0.65)
                    };
                    \addplot[fill=gardqgreen!70] coordinates {
                        (MRR,0.75) (Recall@5,0.87) (NDCG@10,0.91)
                    };
                    \legend{Baseline, GARDQ}
                    \end{axis}
                \end{tikzpicture}
                \end{center}
            \end{block}
        \end{column}
        
        \begin{column}{0.5\textwidth}
            \begin{block}{Améliorations vs Baseline}
                \begin{table}[h]
                \centering
                \footnotesize
                \begin{tabular}{lc}
                \toprule
                \textbf{Métrique} & \textbf{Amélioration} \\
                \midrule
                MRR & \textcolor{gardqgreen}{+78.6\%} \\
                Recall@5 & \textcolor{gardqgreen}{+50.0\%} \\
                NDCG@10 & \textcolor{gardqgreen}{+40.0\%} \\
                BLEU-4 & \textcolor{gardqgreen}{+32.0\%} \\
                \bottomrule
                \end{tabular}
                \end{table}
            \end{block}
        \end{column}
    \end{columns}
    
    \vspace{0.3cm}
    \begin{block}{Impact Opérationnel}
        \begin{center}
        \begin{tabular}{lll}
        \faClockO{} \textbf{Temps de résolution:} -28.6\% &
        \faCheckCircle{} \textbf{Résolution 1er contact:} +35\% &
        \faSmileO{} \textbf{Satisfaction:} +42\%
        \end{tabular}
        \end{center}
    \end{block}
\end{frame}

\begin{frame}{Analyse Qualitative des Résultats}
    \begin{columns}[T]
        \begin{column}{0.6\textwidth}
            \begin{block}{Forces du Système GARDQ}
                \begin{itemize}
                    \item \textbf{Compréhension contextuelle}
                    \begin{itemize}
                        \footnotesize
                        \item Analyse sémantique des requêtes
                        \item Extraction automatique d'entités
                    \end{itemize}
                    \item \textbf{Recherche hybride}
                    \begin{itemize}
                        \footnotesize
                        \item Vectorielle + structure graphe
                        \item Relations multi-niveaux
                    \end{itemize}
                    \item \textbf{Génération intelligente}
                    \begin{itemize}
                        \footnotesize
                        \item Solutions personnalisées
                        \item Traçabilité complète
                    \end{itemize}
                \end{itemize}
            \end{block}
        \end{column}
        
        \begin{column}{0.4\textwidth}
            \begin{block}{ROI Estimé}
                \begin{center}
                \begin{tikzpicture}[scale=0.6]
                    \pie[
                        text=legend,
                        radius=2,
                        color={gardqgreen!70, gardqblue!70, gardqorange!70}
                    ]{
                        30/Réduction coûts,
                        40/Productivité,
                        30/Capitalisation
                    }
                \end{tikzpicture}
                \end{center}
            \end{block}
        \end{column}
    \end{columns}
    
    \vspace{0.3cm}
    \begin{alertblock}{Retour d'Expérience LinkedIn}
        \footnotesize
        "Déployé pendant 6 mois dans l'équipe support, réduction médiane du temps de résolution par ticket de 28.6\%"
    \end{alertblock}
\end{frame}

% Section 7: Conclusion
\section{Conclusion et Perspectives}

\begin{frame}{Contributions et Innovations}
    \begin{columns}[T]
        \begin{column}{0.5\textwidth}
            \begin{block}{Contributions Principales}
                \begin{enumerate}
                    \item \textbf{GraphRAG appliqué aux incidents IT}
                    \begin{itemize}
                        \footnotesize
                        \item Première implémentation complète
                        \item Adaptation du papier LinkedIn
                    \end{itemize}
                    \item \textbf{Pipeline automatisé}
                    \begin{itemize}
                        \footnotesize
                        \item Structuration via LLM
                        \item Construction de graphe scalable
                    \end{itemize}
                    \item \textbf{Méthode d'évaluation}
                    \begin{itemize}
                        \footnotesize
                        \item Métriques adaptées
                        \item Validation empirique
                    \end{itemize}
                \end{enumerate}
            \end{block}
        \end{column}
        
        \begin{column}{0.5\textwidth}
            \begin{block}{Perspectives Futures}
                \begin{itemize}
                    \item \faLightbulb{} \textbf{Apprentissage continu}
                    \begin{itemize}
                        \footnotesize
                        \item Feedback loop
                        \item Mise à jour dynamique
                    \end{itemize}
                    \item \faGlobe{} \textbf{Support multilingue}
                    \begin{itemize}
                        \footnotesize
                        \item Embeddings multilingues
                        \item Traduction automatique
                    \end{itemize}
                    \item \faChartLine{} \textbf{Analyse prédictive}
                    \begin{itemize}
                        \footnotesize
                        \item Détection de patterns
                        \item Prévention proactive
                    \end{itemize}
                \end{itemize}
            \end{block}
        \end{column}
    \end{columns}
    
    \vspace{0.5cm}
    \begin{center}
        \Large
        \textbf{GARDQ transforme les données historiques en intelligence actionnable}
    \end{center}
\end{frame}

\begin{frame}{Merci pour votre attention!}
    \begin{center}
        \Huge \textbf{Questions?}
        
        \vspace{1cm}
        
        \Large
        \faGithub{} Code source disponible\\
        \faBook{} Documentation complète\\
        \faEnvelope{} Contact: Ibrahim Adiao
        
        \vspace{1cm}
        
        \normalsize
        \textit{Référence: LinkedIn SIGIR '24 - arxiv:2404.17723}
    \end{center}
    
    \vspace{0.5cm}
    \begin{block}{Démonstration}
        \centering
        \textbf{Démonstration en direct du système GARDQ disponible}
    \end{block}
\end{frame}

\end{document}